To study these processes, we asked participants to watch sequences of segments from a television show.  After watching different segments, we asked participants to retrodict what they thought likely happened prior to the just-watched segment, predict what would likely happen after the just-watched segment, or recall what they had just watched.  We systematically varied whether participants watched the segments in forward or reverse chronological order, and how many segments preceding or proceeding the target segment they had seen prior to making a response (Fig.~\ref{fig:method}).  We used human annotations and sentence models to evaluate the quality of participants' retrodictions and predictions.  To foreshadow our results, we found that participants were overall better at retrodicting the past than predicting the future.  This appeared to be driven by specific references of past events in conversations.  Taken together, our study provides a systematic summary of the information about the past and future that is contained in the present.





 Inferences about the past and future may be made over a wide range of timescales.  For example, when we encounter a group of friends who are in the middle of a conversation, we can join in by using contextual cues and prior knowledge of those individuals to infer what might have been discussed thus far.

Or when we go on a date with a new potential romantic partner, we might attempt to draw on our observations and prior experiences to infer how the relationship might progress over the long term future.



short timescales (such as when we guess about what some friends have been discussing to join an in-progress conversation, or when we predict where the conversation might go next) or long timescales (such as when we 

(such as when we use contextual cues to join an in-progress conversation) or longer timescales (such as when we go on a date with a new potential romantic partner and attempt to )

which can unfold over short timescales, such as when we use contextual cues to join an in-progress conversation, or over longer timescales, such as when we go on a date with a new potential romantic partner (when we may infer ...). \todo{maybe elaborate a bit more on what things we retrodict? I don't know what'll be the best framing} 

% updated response
Participants had even higher target event hit rates when they made updated responses (versus character-cued responses; OR: 1.92, $Z = 3.71$, $p < 0.001$, CI: 1.36 to 2.72).  Relative to character-cued responses, updated responses were marginally more precise ($b = 0.03$, $t(15.5) = 2.03$, $p = 0.06$, CI: 0.00 to 0.06), but were not reliably more similar across individuals ($b = 0.04$, $t(16.9) = 1.68$, $p = 0.11$, CI: -0.01 to 0.09).  Relative to updated responses, participants' recalls showed higher target event hit rates (OR = 11.35, $Z = 9.74$, $p < 0.001$, CI: 6.96 to 18.50) and more convergent across individuals ($b = 0.16$, $t(18.3) = 6.93$, $p < 0.001$, CI: 0.11 to 0.21). 

In updated retrodiction (\textit{c-RP}) and updated prediction (\textit{c-PR}) conditions, participants additionally watched one segment earlier or one segment later. Thus, when comparing these two conditions, as both segment n-1 and n+1 had been watched, we were comparing the contributions of segment n+2 to 11 with that of segment 1 to n-2 (Fig.~\ref{fig:methods}, \textit{Data overview}).
We did not find statistically reliable differences between updated retrodiction ('c-RP') and updated prediction ('c-PR') conditions in target event hit rates (OR = 1.12,~$Z = 0.77,~p = 0.44$,~CI: 0.83 to 1.51), precision ($b = 0.01,~t(15.5) = 0.79,~p = 0.44$,~CI: -0.02 to 0.04), or convergence ($b = 0.00,~t(14.5) = -0.16,~p = 0.88$,~CI: -0.08 to 0.07). 

For the updated retrodiction and prediction tasks, participants were also cued with the characters (as in the character-cued retrodiction and prediction), but their previous responses were not provided. 

% retrodicing past events one's own life
\todo{reframe this point?} However, remembering the past and predicting the future do seem to share some common mechanisms~\citep[e.g.,][]{SchaEtal07}.  For example, some of our past experiences may not be encoded with perfect fidelity, and reconstructing a ``missing'' past experience can be accomplished using processes similar to how we might estimate a future experience.  This is especially true when we estimate the pasts and futures of \textit{other} people's lives, e.g., when we are equally ignorant of the past and future relative to a given moment.

% adjacent references
We found that conversational references also caused cascading effect such that the hit rates of events adjacent to referenced events were also increased. Referenced events could serve as anchor points that turned the original extrapolation task into an interpolation task, that was, filling in the missing events between the current event and the referenced events. While for any individual, direct observations of the world are limited, memory sharing in conversations provides people with much more observations of the world. In this sense, conversation references could be viewed as expanding the 'interpolation zone' \citep{HassEtal20}. 

% no reverse asymmetry
Just as references provided additional information of the referenced events, they may also provide 
There were more references to past events also meant there would be more future events that would make references to past events. In other words, what is happening now is more likely to be talked about in the future, than in the past. Thus, participants might simply predict that 'characters will talk about what is happening now' as a future event. However, we found that for events that made references to other events, hit rates were not boosted. One possible extrapolation was that, referenced events were almost always compressed. We could talk about past events in highly compressed manner, covering lots of information in relatively short amount of time, but 'we will talk about what is happening now in the future' only contains limited amount of information about the future. Plus, not all events happened will be referred to in the future. Thus, the boost of information of a reference is inherently asymmetric.

% the potential of the paradigm
While studies have shown that the processing current inputs are affected by past stimulus of different timescales \citep{HassEtal15, LernEtal11}, as measured with inter-subject correlations of brain activities, the retrodiction and prediction tasks used here could measure how retrodicting past inputs and predicting future inputs are affected by different amount of information available. The current approach could also uncover how events are related in terms of shared information, and give insights into how we form long-term links during event perception and memory \citep{HeusEtal21, Mann21a, ChanEtal21b, CohnEtal21}.